
\documentclass[a4paper,12pt]{article}

\usepackage{listings}
\usepackage{../lab}

\begin{document}

\labtitle{1}{1}

\labtheme{Изучение основ объектно-ориентированного программирования на
  языке программирования С++}

\begin{itemize}
\item понятие класса и объекта;
\item поля и методы класса;
\item статические методы класса;
\item определение методов внутри и вне класса;
\item принцип инкапсуляции;
\item управление доступом (области видимости);
\item конструктор и деструктор;
\item указатель \lstinline|this|;
\item ссылка на объект;
\item использование спецификаторы \lstinline|const| в возвращаемом
  типе, в типе аргументов и после метода класса;
\item конструктор по умолчанию;
\item конструктор копирования;
\item спецификатор \lstinline|mutable| для поля класса;
\item динамическое выделение памяти под объекты (операторы
  \lstinline|new| и \lstinline|delete|);
\item ссылка на себя;
\item перегрузка операторов \lstinline|+|, \lstinline|-|,
  \lstinline|*|, \lstinline|/|, \lstinline|=|, \lstinline|==|,
  \lstinline|!=|, \lstinline|>|, \lstinline|>=|. \lstinline|<|,
  \lstinline|<=| и т.д.;
\item перегрузка потоковых операторов (\lstinline|<<|,
  \lstinline|>>|);
\item операторы-члены и не-члены;
\item \lstinline|std::string| (строки в стиле С++).
\item документирование кода при помощи \verb|doxygen|:
  \begin{itemize}
  \item краткое и полное описание;
  \item тэг;
  \item тэги \verb|file|, \verb|author|, \verb|brief|, \verb|details|,
    \verb|param|, \verb|return|, \verb|see|;
  \item генерация документации.
  \end{itemize}
\end{itemize}

\section{Задание}
\label{sec:Task}

Реализовать на языке \verb|С++| один из нижеперечисленных вариантов и
написать и сгенерировать документацию при помощи \verb|doxygen|. Для
возможности тестирования классов написать тестовую программу с меню
или набор unit-тестов. В случае написания unit-тестов необходимо
проверить не менее 30 тестов случаев с использованием библиотеки
\verb|UnitTest++|.

Каждый из реализованных классов должен иметь следующие свойства:

\begin{itemize}
\item инкапсуляция;
\item отделение консольного пользовательского интерфейса программы от
  реализации класса;
\item конструктор копирования и оператор \lstinline|=|, если это
  уместно;
\item деструктор, если это необходимо;
\item операторы для сравнения (операторы \lstinline|==|,
  \lstinline|!=|);
\item операторы для работы с потоками ввода (\lstinline|>>|) и вывода
  (\lstinline|<<|);
\item взаимная независимость класса и пользовательского интерфейса,
  использующего его;
\item отделение объявления класса в \texttt{h}-файл, а реализации ---
  в \texttt{cpp}-файл.
\end{itemize}

\subsection{Вектор}

Описать класс для работы с вектором, который задается координатами
концов в трехмерном пространстве.  Класс должен реализовывать
следующие возможности:

\begin{itemize}
\item получение координат концов вектора;
\item вычисление длины вектора;
\item сложение двух векторов (операторы \lstinline|+|,
  \lstinline|+=|);
\item вычитание двух векторов (операторы \lstinline|-|,
  \lstinline|-=|);
\item произведение двух векторов (операторы \lstinline|*|,
  \lstinline|*=|);
\item произведение вектора на число (операторы \lstinline|*|,
  \lstinline|*=|);
\item деление двух векторов (оператор \lstinline|/|, \lstinline|/=|);
\item определение косинуса между двумя векторами (оператор
  \lstinline|^|);
\item операторы для сравнения двух векторов (операторы \lstinline|>|,
  \lstinline|>=|, \lstinline|<|, \lstinline|<=|);
\end{itemize}

\subsection{Англо-русский словарь}

Описать класс, реализующий англо-русский словарь в виде бинарного
дерева (в узлах хранится пара слов, причем английское слово является
ключом). Класс должен реализовывать следующие возможности:

\begin{itemize}
\item добавление нового английского слова и перевода для него
  (оператор \lstinline|+=| для работы как со строками в стиле С -
  \lstinline|char *|, так и в стиле С++ - \lstinline|std::string|);
\item удаление существующего английского слова из словаря (оператор
  \lstinline|-=|);
\item поиск перевода английского слова (оператор \lstinline|[]|);
\item замена перевода английского слова (оператор \lstinline|[]|);
\item определение количества слов в словаре;
\item возможность загрузки словаря из файла;
\end{itemize}

\subsection{Прямоугольник}

Описать класс прямоугольника со сторонами, параллельными осям
координат. Вершины прямоугольников имеют должны иметь целочисленные
координаты. Класс должен реализовывать следующие возможности:

\begin{itemize}
\item получение координат вершин;
\item перемещение;
\item изменение размера;
\item увелечение размера на единицу по каждой из осей (оператор пре- и
  постинкремента \lstinline|++|);
\item уменьшение размера на единицу по каждой из осей (оператор пре- и
  постдекремента \lstinline|--|);
\item построение наименьшего прямоугольника, содержащего два заданных
  прямоугольника (оператор \lstinline|+|);
\item построение наименьшего прямоугольника, содержащего два заданных
  прямоугольника, с присваиванием (оператор \lstinline|+=|);
\item построение прямоугольника, являющегося общей частью
  (пересечением) двух прямоугольников (оператор \lstinline|-|);
\item построение прямоугольника, являющегося общей частью
  (пересечением) двух прямоугольников, с присванием (оператор
  \lstinline|-=|);
\end{itemize}

\subsection{Теория множеств}

Описать класс "Множество".
Класс должен реализовывать следующие возможности:

\begin{itemize}
\item элементом множества может быть другое множество;
\item проверка на пустое множество;
\item добавление элемента;
\item удаление элемента;
\item определение мощности множества;
\item проверка принадлежности элемента множеству (оператор
  \lstinline|[]|);
\item объединение двух множеств (операторы \lstinline|+| и
  \lstinline|+=|);
\item пересечение двух множеств (операторы \lstinline|*| и
  \lstinline|*=|);
\item разность двух множеств (операторы \lstinline|-| и
  \lstinline|-=|);
\item построение булеана данного множества;
\end{itemize}

\subsubsection{Множество}

Описать класс "Неориентированное канторовское множество".  Класс
должен реализовывать следующие возможности:

\begin{itemize}
\item формирование множества из строки, как из \lstinline|char *|, так
  и из \lstinline|std::string|; например:
\begin{verbatim}
{a, b, c, {a, b}, {}, {a, {c}}}.
\end{verbatim}
\end{itemize}

\subsubsection{Мультимножество}

Описать класс "Неориентированное мультимножество". Класс должен
реализовывать следующие возможности:

\begin{itemize}
\item формирование множества из строки, как из \lstinline|char *|, так
  и из \lstinline|std::string|; например:
\begin{verbatim}
{a, a, c, {a, b, b}, {}, {a, {c, c}}}).
\end{verbatim}
\end{itemize}

\subsection{Игры}

Все программы-игры должны иметь консольный интерфейс и меню для игры.

\subsubsection{Крестики-нолики}

Описать класс, реализующую игру "Крестики-нолики" (поле произвольного размера) между двумя игроками.
Класс должен реализовывать следующие возможности:

\begin{itemize}
\item проверки возможности установки крестика/нолика в указанной
  позиции;
\item получение значения указанной позиции (оператор []);
\item установка значения указанной позиции (оператор []);
\item проверка победы одного из игроков;
\end{itemize}

\subsubsection{Пятнашки}

Описать класс, реализующий игру-головоломку "Пятнашки". Начальное
размещение номеров — случайное. Реализовать методы для осуществления
перестановки клеток, для проверки правильной расстановки клеток.
Класс должен реализовывать следующие возможности:

\begin{itemize}
\item cлучайное начальное размещение номеров;
\item перестановка клеток;
\item получения значения клетки (оператор \lstinline|[]|);
\item проверка правильной расстановки клеток.
\end{itemize}

\subsubsection{Кубик Рубика}

Описать класс, реализующий игру-головоломку "Кубик Рубика".  Класс
должен реализовывать следующие возможности:

\begin{itemize}
\item случайное начальное размещение цветов;
\item загрузка начального размещения цветов из файла;
\item поворот грани кубика;
\item проверка правильной расстановки цветных клеток;
\end{itemize}

\subsection{Математика}

\subsubsection{Натуральная дробь со знаком}

Описать класс, реализующий тип данных "Натуральная дробь со
знаком". Натуральная дробь всегда должна хранится в сокращенном виде.
Класс должен реализовывать следующие возможности:

\begin{itemize}
\item получение определителя, знаменателя и целой части;
\item сложение двух натуральных дробей (операторы \lstinline|+|,
  \lstinline|+=|);
\item сложение натуральной дроби с целым (операторы \lstinline|+|,
  \lstinline|+=|);
\item вычитание двух натуральных дробей (операторы \lstinline|-|,
  \lstinline|-=|);
\item вычитание из натуральной дроби с целого (операторы \lstinline|-|,
  \lstinline|-=|);
\item произведение двух натуральных дробей (операторы \lstinline|*|,
  \lstinline|*=|);
\item произведение натуральной дроби и целого (операторы \lstinline|*|,
  \lstinline|*=|);
\item деление двух натуральных дробей (операторы \lstinline|/|,
  \lstinline|/=|);
\item деление натуральной дроби на целое (операторы \lstinline|/|,
  \lstinline|/=|);
\item операторы пре- и постинкремента, пре- и постдекремента
  (операторы \lstinline|++|, \lstinline|--|);
\item сравнение двух натуральных дробей и натуральной дроби с целым
  (операторы \lstinline|>|, \lstinline|>=|, \lstinline|<|,
  \lstinline|<=|);
\item приведение к \lstinline|double|;
\end{itemize}

\subsubsection{Многочлен от одной переменной}

Описать класс многочлена от одной переменной, задаваемых степенью
многочлена и массивом коэффициентов. Класс должен реализовывать
следующие возможности:

\begin{itemize}
\item получение значений коэффициентов (оператор \lstinline|[]|);
\item вычисление значения многочлена для заданного аргумента (оператор
  \lstinline|()|);
\item сложение двух многочленов (операторы \lstinline|+|,
  \lstinline|+=|);
\item вычитание двух многочленов (операторы \lstinline|-|,
  \lstinline|-=|);
\item произведение двух многочленов (операторы \lstinline|*|,
  \lstinline|*=|);
\item деление двух многочленов (операторы \lstinline|/|,
  \lstinline|/=|);
\end{itemize}

\subsubsection{Длинное целое со знаком}

Описать класс, реализующий тип данных "длинное целое со знаком" и
работу с ними. Длинна целого числа должна быть неограниченной. Класс
должен реализовывать следующие возможности:

\begin{itemize}
\item оператор преобразования длинного целого к целому;
\item сложение двух длинных целых (операторы \lstinline|+|,
  \lstinline|+=|)
\item сложение длинного целого с целым (оператор \lstinline|+|,
  \lstinline|+=|);
\item вычитание двух длинных целых (операторы \lstinline|-|,
  \lstinline|-=|)
\item вычитание из длинного целого целого (оператор \lstinline|-|,
  \lstinline|-=|);
\item произведение двух длинных целых (операторы \lstinline|*|,
  \lstinline|*=|)
\item произведение длинного целого и целого (оператор \lstinline|*|,
  \lstinline|*=|);
\item деление двух длинных целых (операторы \lstinline|/|,
  \lstinline|/=|)
\item деление длинного целого на целое (оператор \lstinline|/|,
  \lstinline|/=|);
\item пре- и постинкремент (\lstinline|++|), пре- и постдекремент
  (\lstinline|--|);
\item сравнение двух длинных целых (операторы \lstinline|>|,
  \lstinline|>=|, \lstinline|<|, \lstinline|<=|);
\item сравнение длинного целого с целым (операторы \lstinline|>|,
  \lstinline|>=|, \lstinline|<|, \lstinline|<=|);
\end{itemize}

\subsection{Теория матриц}

Описать класс, реализующий тип данных "Вещественная матрица".  Класс
должен реализовывать следующие возможности:

\begin{itemize}
\item матрица произвольного размера с динамическим выделением памяти;
\item пре- и постинкремент (\lstinline|++|), пре- и постдекремент
  (\lstinline|--|);
\end{itemize}

\subsubsection{Вещественная матрица}

Класс должен реализовывать следующие дополнительные возможности:

\begin{itemize}
\item изменение числа строк и числа столбцов;
\item загрузка матрицы из файла;
\item извлечение подматрицы заданного размера;
\item проверка типа матрицы (квадратная, диагональная, нулевая,
  единичная, симметрическая, верхняя треугольная, нижняя треугольная);
\item транспонированние матрицы;
\end{itemize}

\subsubsection{Вещественная матрица (усложненный)}

Класс должен реализовывать следующие дополнительные возможности:

\begin{itemize}
\item сложение двух матриц (операторы \lstinline|+|, \lstinline|+=|);
\item сложение матрицы с числом (операторы \lstinline|+|,
  \lstinline|+=|);
\item вычитание двух матриц (операторы \lstinline|-|, \lstinline|-=|);
\item вычитание из матрицы числа (операторы \lstinline|-|,
  \lstinline|-=|);
\item произведение двух матриц (оператор \lstinline|*|);
\item произведение матрицы на число (операторы \lstinline|*|,
  \lstinline|*=|);
\item деление матрицы на число (операторы \lstinline|/|,
  \lstinline|/=|);
\item возведение матрицы в степень (оператор \lstinline|^|,
  \lstinline|^=|);
\item вычисление детерминанта;
\item вычисление нормы;
\end{itemize}

\subsection{Абстрактные машины}

Описать классы, реализующие абстрактную машину, ее ленту, программу и
правила, каретку, алфавит, строку. Набор классов зависит от вида
абстрактной машины. Количество классов может варьироваться, но
минимально должно быть три класса.  Классы должны реализовывать
следующие возможности:

\begin{itemize}
\item загрузка программы из потока ввода;
\item загрузка состояние ленты из потока;
\item добавления/удаления/просмотр правил;
\item задания/изменения значений на ленте;
\item осуществление шага и интерпретации всей программы (можно при
  помощи опрераторов инкремента и декремента);
\end{itemize}

Написать программу, которая принимает в качестве аргумента командной
строки путь к файлу, который содержит начальное абстрактной машины и
программу для интепретации. После запуска программа считывает
содержимое этого файла и осуществляет интерпретацию правил. Если в
командной строке был указан аргумент \verb|-log|, тогда после
выполнения каждого правила на консоль должна выводится информация о
состоянии абстрактной машины.

\subsubsection{Машина Поста}

Описать классы, реализующие машину Поста.

\subsubsection{Машина Тьюринга}

Описать классы, реализующие машину Тьюринга.

\subsubsection{Нормальные алгорифмы Маркова}

Описать классы, реализующие
\href{http://ru.wikipedia.org/wiki/%D0%9D%D0%BE%D1%80%D0%BC%D0%B0%D0%BB%D1%8C%D0%BD%D1%8B%D0%B9_%D0%B0%D0%BB%D0%B3%D0%BE%D1%80%D0%B8%D1%82%D0%BCe_%D0%9C%D0%B0%D1%80%D0%BA%D0%BE%D0%B2%D0%B0}{нормальные алгорифмы Маркова}.

\section{Литература}
\label{sec:Literature}

\begin{itemize}
\item Х. Дейтел "Как программировать на С++" 5-ое издание 2008 года

  Для выполнения лабораторной работы необходимо прочитать главы 3, 9,
  10, 11. Ссылочный тип и спецификатор const и mutable не описаны в
  этих главах. Эта книга рекомендуется.

\item Б. Страуструп "Язык программирования С++"

  Для выполнения лабораторной работы необходимо прочитать главы 5 и
  6. Ссылочный тип и спецификатор const описаны в более ранних
  главах. Для новичка эта книга сложновата.

\item С. Макконнелл "Совершенный код"

  В этой книге необходимо прочитать главу 22, которая связана с
  тестированием ПО, которая проводится разработчиком.

\item \href{http://doxygenorg.ru/old/}{Doxygen на русском}

\end{itemize}

\section{Часто встречающиеся ошибки}

Студенты забывают использовать ссылки C++ \lstinline|&| для избежания
копирования объектов. Например:
\begin{lstlisting}[texcl]
// Накладные расходы на копирование объекта
void f(std::string str);

// Передача по ссылке без копирования.
// В данном случае не забываем про \verb|const|,
// чтобы нельзя было изменить объект из функции
void f(const std::string &str);
\end{lstlisting}

Отсутствие спецификатора \lstinline|const| в местах, где он необходим
по смысле. Необходимо как можно чаще его использовать в возвращаемых
типах, в типах аргументов и для задания константных методов.

Это делает код строже, позволяет лучше специфицировать интерфейс для
внешнего пользователя и избежать ошибок (компилятор будет ругаться на
попытки изменить \lstinline|const|-объект).

Отсутствие понимания разницы между оператором \lstinline|+| и
\lstinline|+=| (с присваиванием) (тоже самое и для других похожих
операторов). Внимательно изучите возвращаемые типы этих операторов и
разберитесь с понятием "Ссылка на себя".

Студенты не могут ответить на вопрос "В каких случаях нужен
конструктор копирования?".

\end{document}


