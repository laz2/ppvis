
\documentclass[a4paper,12pt]{article}

\usepackage{../lab}

\begin{document}

\labtitle{2}{1}

\labtheme{Изучение возможных принципов наследования при
  объектно-ориентированного программирования с использованием языка
  С++}

\tableofcontents

\section{Задание}
\label{sec:Task}

Каждому студенту требуется выбрать предметную область и согласовать её
с преподавателем. В рамках выбранной предметной области следует
построить иерархию классов с применением следующих концепций ООП:

\begin{itemize}
\item закрытое
\item открытое
\item защищённое наследования
\item виртуальные методы
\item разрешение имён с помощью \lstinline|using|-директивы
\item множественное наследование
\item виртуальное наследование.
\end{itemize}

\section{Варианты}
\label{sec:Variants}

\begin{enumerate}
\item Зоопарк, иерархия классов для животных
\item Плоские геометрические фигуры
\item Объёмные геометрические фигуры
\item Административно-территориальное деление и населённые пункт
\item Университет
\item Научные дисциплины
\item Вооружение
\item Мебель
\item Одежда
\item Произведения искусства
\item Транспортные средства
\item Книги
\item Кулинарные блюда и рецепты
\end{enumerate}

\section{Литература}
\label{sec:Literature}

\begin{itemize}
\item Х. Дейтел <<Как программировать на С++>> 5-ое издание 2008 года

  Для выполнения лабораторной работы необходимо прочитать главу 12
  "Объектно-ориентированное программирование: Наследование" и главу 13
  "Объектно-ориентированное программирование: Полиморфизм".
\item Б. Страуструп <<Язык программирования С++>>
  
  Для выполнения лабораторной работы необходимо прочитать главу 12
  "Производные классы".
\end{itemize}
\end{document}


